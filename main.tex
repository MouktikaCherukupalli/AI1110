\documentclass[12pt,two column]{article}
\usepackage[utf8]{inputenc}
\usepackage{polynom}
\usepackage{times}
\usepackage{amsmath}
\usepackage{amssymb}
\title{Assignment 1}
\author{ai21btech11007 }
\date{March 2022}
\polyset{%
  style=A,
  }

\begin{document}
\maketitle

\textbf{question}\\
Using the factor theorem show that x-2 is a factor of $x^3+x^2-4x-4$.hence factorise the polynomial completely

\textbf{solution: }\\
By the factor theorem if $f(a)=0$,then $x-a$ will be factor of f(x)
\\let the given polynomial be f(x)
\begin{align}
  f(x) &= x^3+x^2-4x-4\\  
  f(2) &= 2^3+2^2-4\times2-4\\
\implies
f(2) &= 0
\end{align}
   so,x-2 is a factor of f(x),now to factorise f(x) 
\[
 \polylongdiv{x^3+x^2-4x-4}{x-2}
 \]
  we get $x^2+3x+2$
  which is a quadratic expression so we can factorise it further by finding it's roots
 \begin{align}
     x &= \frac{-b \pm \sqrt{b^2-4\times a\times c}}{2\times a}
 \end{align}
 here,
 \begin{align}
  b &=3\\
  a &=1\\
  c &=2
 \end{align}
 so roots would be
 \begin{align}
   x &= \frac{-3 \pm \sqrt{3^2-4\times 1\times 2}}{2\times 1} \\
   x &= -1 and -2
 \end{align}
 so -1 and -2 are the roots
\begin{align}
\implies
\textbf{other two factors are $x+1$ and $x+2$}
\end{align}

$\therefore$ the final factors of f(x) are $x+1$,$x+2$ and $x-2$
\begin{align}
f(x)&=( x+1)\times( x-2)\times( x+2)
\end{align}
\end{document}
