\documentclass[12pt,two column]{article}
\usepackage[utf8]{inputenc}
\usepackage{polynom}
\usepackage{times}
\usepackage{amsmath}
\title{Assignment 1}
\author{ai21btech11007 }
\date{March 2022}
\polyset{%
  style=C,
  }

\begin{document}

\maketitle
\textbf{question}\\
Using the factor theorem show that x-2 is a factor of $x^3+x^2-4x-4$.hence factorise the polynomial completely
\\

\textbf{solution: }\\
By the factor theorem if $f(a)=0$,then $x-a$ will be factor of f(x)
\\let the given polynomial be f(x)
\begin{align}
  f(x) &= x^3+x^2-4x-4\\  
  f(2) &= 2^3+2^2-4*2-4\\
\implies
f(2) &= 0
\end{align}
   \\so,x-2 is a factor of f(x),now to factorise f(x) 
  \[
    \polylongdiv{x^3+x^2-4x-4}{x-2}
  \]
  \\we get $x^2+3x+2$
  \\which is a quadratic expression so we can factorise it further by finding it's roots
  \\roots are 
  \[ \frac{-b \pm \sqrt{b^2-4\times a\times c}}{2\times a} \]
  \\here { $b= 3$,
           $a=1$,
           $c=2$ }
 \\so roots would be
   \[ \frac{-3 \pm \sqrt{3^2-4\times 1\times 2}}{2\times 1} \]
   \\ -1 and -2 are roots
   \\so other two factors are
  \\$x+1$ and $x+2$
  \\the final factors are 
  \\$x+1$,$x-2$ and $x+2$

$f(x)=( x+1)\times( x-2)\times( x+2)$

\end{document}
